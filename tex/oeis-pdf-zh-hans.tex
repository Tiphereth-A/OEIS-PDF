\documentclass{oeiscommon}
\usepackage{ctex}


\begin{document}

\thispagestyle{empty}
\newgeometry{vmargin=3cm}

本文档包含了 \href{\oeis}{The On-Line Encyclopedia of Integer Sequences{\(^{\textregistered}\)} (OEIS{\(^{\textregistered}\)})} 中的部分数列.

\paragraph{排序方式} 令 \(a\) 为删去原序列前导 \(0\) 后的序列, \(b\) 为删去 \(a\) 的前导 \(1\) 后得到的序列, 则:

\begin{itemize}
    \item 第一关键字是 \(b\)
    \item 第二关键字是 \(a\) 的前导 \(1\) 长度 \(c_1\)
    \item 第三关键字是原序列的前导 \(0\) 长度 \(c_0\)
    \item 第四关键字是 OEIS 中对应的 ID (JSON 中的 \verb|number|, 不是 \verb|id|)
\end{itemize}

序列的前导 \(x\) 定义为: 从序列的首个元素开始依次向后, 直到遇到第一个非 \(x\) 元素为止这一段子序列, 如:

\begin{itemize}
    \item 序列 {\ttfamily 0,0,1,0,1,5} 的前导 \(0\) 是 {\ttfamily 0,0}
    \item 序列 {\ttfamily 1,1,1,0,1,2,1,0} 的前导 \(1\) 是 {\ttfamily 1,1,1}
\end{itemize}

\paragraph{格式} \textbf{OEIS ID}\(\langle\)Offset\(\rangle\) \{\(c_0\),\(c_1\)\}\(b\) \textit{Description} \(\dagger\) Formula 1  \(\dagger\) Formula 2 ... \href{\oeis}{\P}

若 \(c_0\) 或 \(c_1\) 为 \(0\) 则省略 \(c_0\) 或 \(c_1\), 若同时为 \(0\), 则省略 \{\(c_0\),\(c_1\)\} 部分

\(b\) 对相邻的相同元素进行了压缩:

{\ttfamily 2,2,2,2,2,3,3,3,4,4,5,10,10} \(\implies\) {\ttfamily 2:5,3:3,4:2,5,10:2}

为控制篇幅, \(b\) 只保留压缩后的前 \(10\) 项

公式部分按如下规则加以简化:

\begin{itemize}
    \item \(\dagger\) a(n) = xxx \(\implies\) \(\ddagger\) xxx 
    \item \(\dagger\) Recurrence xxx \(\implies\) \(\circlearrowleft\) xxx 
    \item \(\dagger\) G.f. xxx \(\implies\) \(\flat\) xxx 
    \item \(\dagger\) E.g.f. xxx \(\implies\) \(\natural\) xxx 
    \item \(\dagger\) Dirichlet g.f. xxx \(\implies\) \(\diamond\) xxx 
\end{itemize}

\paragraph{生成脚本} \href{\repo}{\repo}

\textbf{注意}:本文档正文内容版权归 OEIS{\(^{\textregistered}\)} 所有,请遵循相关法律。

\paragraph{OEIS{\(^{\textregistered}\)} 相关法律文件链接} \href{\oeislegaldoc}{\oeislegaldoc}

\paragraph{以下为正文}

\newpage
\linespread{1.0}
\newgeometry{includefoot,inner=1cm,outer=.5cm,vmargin=.5cm,footskip=12.0pt}
\twocolumn
\footnotesize
\setcounter{page}{1}
\pagestyle{fancy}
\fancyhf{}
\fancyfoot[LE,RO]{\thepage}

\input{oeis.tex}

\newgeometry{includefoot,hmargin=1cm,vmargin=.5cm,footskip=12.0pt}
\footnotesize
\setcounter{page}{1}
\pagestyle{fancy}
\fancyhf{}
\fancyfoot[C]{\thepage}
\printindex

\end{document}
